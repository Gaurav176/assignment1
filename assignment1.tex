
\let\negmedspace\undefined
\let\negthickspace\undefined
\documentclass[journal,12pt,twocolumn]{IEEEtran}
%\documentclass[conference]{IEEEtran}
%\IEEEoverridecommandlockouts
% The preceding line is only needed to identify funding in the first footnote. If that is unneeded, please comment it out.
\usepackage{cite}
\usepackage{amsmath,amssymb,amsfonts,amsthm}
\usepackage{algorithmic}
\usepackage{graphicx}
\usepackage{textcomp}
\usepackage{xcolor}
\usepackage{txfonts}
\usepackage{listings}
\usepackage{enumitem}
\usepackage{mathtools}
\usepackage{gensymb}
\usepackage[breaklinks=true]{hyperref}
\usepackage{tkz-euclide} % loads  TikZ and tkz-base
\usepackage{listings}
%
%\usepackage{setspace}
%\usepackage{gensymb}
%\doublespacing
%\singlespacing

%\usepackage{graphicx}
%\usepackage{amssymb}
%\usepackage{relsize}
%\usepackage[cmex10]{amsmath}
%\usepackage{amsthm}
%\interdisplaylinepenalty=2500
%\savesymbol{iint}
%\usepackage{txfonts}
%\restoresymbol{TXF}{iint}
%\usepackage{wasysym}
%\usepackage{amsthm}
%\usepackage{iithtlc}
%\usepackage{mathrsfs}
%\usepackage{txfonts}
%\usepackage{stfloats}
%\usepackage{bm}
%\usepackage{cite}
%\usepackage{cases}
%\usepackage{subfig}
%\usepackage{xtab}
%\usepackage{longtable}
%\usepackage{multirow}
%\usepackage{algorithm}
%\usepackage{algpseudocode}
%\usepackage{enumitem}
%\usepackage{mathtools}
%\usepackage{tikz}
%\usepackage{circuitikz}
%\usepackage{verbatim}
%\usepackage{tfrupee}
%\usepackage{stmaryrd}
%\usetkzobj{all}
%    \usepackage{color}                                            %%
%    \usepackage{array}                                            %%
%    \usepackage{longtable}                                        %%
%    \usepackage{calc}                                             %%
%    \usepackage{multirow}                                         %%
%    \usepackage{hhline}                                           %%
%    \usepackage{ifthen}                                           %%
  %optionally (for landscape tables embedded in another document): %%
%    \usepackage{lscape}     
%\usepackage{multicol}
%\usepackage{chngcntr}
%\usepackage{enumerate}

%\usepackage{wasysym}
%\newcounter{MYtempeqncnt}
\DeclareMathOperator*{\Res}{Res}
%\renewcommand{\baselinestretch}{2}
\renewcommand\thesection{\arabic{section}}
\renewcommand\thesubsection{\thesection.\arabic{subsection}}
\renewcommand\thesubsubsection{\thesubsection.\arabic{subsubsection}}

\renewcommand\thesectiondis{\arabic{section}}
\renewcommand\thesubsectiondis{\thesectiondis.\arabic{subsection}}
\renewcommand\thesubsubsectiondis{\thesubsectiondis.\arabic{subsubsection}}

\newcommand\Mycomb[2][^n]{\prescript{#1\mkern-0.5mu}{}C_{#2}}


% correct bad hyphenation here
\hyphenation{op-tical net-works semi-conduc-tor}
\def\inputGnumericTable{}                                 %%

\lstset{
%language=C,
frame=single, 
breaklines=true,
columns=fullflexible
}
%\lstset{
%language=tex,
%frame=single, 
%breaklines=true
%}
\usepackage{amsmath}
\begin{document}
%


\newtheorem{theorem}{Theorem}[section]
\newtheorem{problem}{Problem}
\newtheorem{proposition}{Proposition}[section]
\newtheorem{lemma}{Lemma}[section]
\newtheorem{corollary}[theorem]{Corollary}
\newtheorem{example}{Example}[section]
\newtheorem{definition}[problem]{Definition}
%\newtheorem{thm}{Theorem}[section] 
%\newtheorem{defn}[thm]{Definition}
%\newtheorem{algorithm}{Algorithm}[section]
%\newtheorem{cor}{Corollary}
\newcommand{\BEQA}{\begin{eqnarray}}
\newcommand{\EEQA}{\end{eqnarray}}
\newcommand{\define}{\stackrel{\triangle}{=}}

\bibliographystyle{IEEEtran}
%\bibliographystyle{ieeetr}


\providecommand{\mbf}{\mathbf}
\providecommand{\pr}[1]{\ensuremath{\Pr\left(#1\right)}}
\providecommand{\qfunc}[1]{\ensuremath{Q\left(#1\right)}}
\providecommand{\sbrak}[1]{\ensuremath{{}\left[#1\right]}}
\providecommand{\lsbrak}[1]{\ensuremath{{}\left[#1\right.}}
\providecommand{\rsbrak}[1]{\ensuremath{{}\left.#1\right]}}
\providecommand{\brak}[1]{\ensuremath{\left(#1\right)}}
\providecommand{\lbrak}[1]{\ensuremath{\left(#1\right.}}
\providecommand{\rbrak}[1]{\ensuremath{\left.#1\right)}}
\providecommand{\cbrak}[1]{\ensuremath{\left\{#1\right\}}}
\providecommand{\lcbrak}[1]{\ensuremath{\left\{#1\right.}}
\providecommand{\rcbrak}[1]{\ensuremath{\left.#1\right\}}}
\theoremstyle{remark}
\newtheorem{rem}{Remark}
\newcommand{\sgn}{\mathop{\mathrm{sgn}}}
\providecommand{\abs}[1]{\left\vert#1\right\vert}
\providecommand{\res}[1]{\Res\displaylimits_{#1}} 
\providecommand{\norm}[1]{\left\lVert#1\right\rVert}
%\providecommand{\norm}[1]{\lVert#1\rVert}
\providecommand{\mtx}[1]{\mathbf{#1}}
\providecommand{\mean}[1]{E\left[ #1 \right]}
\providecommand{\fourier}{\overset{\mathcal{F}}{ \rightleftharpoons}}
%\providecommand{\hilbert}{\overset{\mathcal{H}}{ \rightleftharpoons}}
\providecommand{\system}{\overset{\mathcal{H}}{ \longleftrightarrow}}
	%\newcommand{\solution}[2]{\textbf{Solution:}{#1}}
\newcommand{\solution}{\noindent \textbf{Solution: }}
\newcommand{\cosec}{\,\text{cosec}\,}
\providecommand{\dec}[2]{\ensuremath{\overset{#1}{\underset{#2}{\gtrless}}}}
\newcommand{\myvec}[1]{\ensuremath{\begin{pmatrix}#1\end{pmatrix}}}
\newcommand{\mydet}[1]{\ensuremath{\begin{vmatrix}#1\end{vmatrix}}}
%\numberwithin{equation}{section}
%\numberwithin{equation}{subsection}
%\numberwithin{problem}{section}
%\numberwithin{definition}{section}
%\makeatletter
%\@addtoreset{figure}{problem}
%\makeatother

%\let\StandardTheFigure\thefigure
\let\vec\mathbf

\title{Assignment 1 \\ \Large AI1110: Probability and Random Variables \\ \Large Indian Institute of Technology Hyderabad}
\author{Gaurav Choudekar \\ \normalsize CS22BTECH11015}

	% The title
        \maketitle	
	% The question
	\textbf{Question :}
	A fair coin is tossed four times, and a person win  Rs 1 for each head and loseRs 1.5 for each tail that turns up.From the sample space calculate how miany different amounts of money you canhave after four tosses and the probability of having each of these amounts.
        
	% The solution
	\textbf{Solution.}

	The Sample space of for tosses is 
	S = \{TTTT,TTTH,TTHT,THTT,HTTT, TTHH,THTH,THHT,HTHT,HHTT,HTTH,THHH, HTHH,HHTH,HHHT,HHHH\}\\
	After 4 tosses he can have 5 different amounts
        \begin{enumerate}
	\item	4 heads \& 0 tails - 1*4-1.5*0=4
	\item	3 heads \& 1 tails - 1*3-1.5*1=1.5
	\item	2 heads \& 2 tails - 1*2-1.5*2=-1
	\item   1 head  \& 3 tails - 1*1-1.5*3=-3.5
	\item   0 heads \& 4 tails - 1*0-1.5*4=-6       
	\end{enumerate}

	Let us solve this by binomial random variable.
        here,random variable is a function assigning values of number of heads in 4 tosses to amount won after 4 tosses
       
        we know,for binomial random variable,probability is given by \pr{x,n,P}\\
        x=number of successes(here number of heads\\
        n=number of trials(here 4)\\
        P=probability of success(probability of head=0.5)\\
        \pr{x,n,P}= ${\Mycomb[n]{x}}$*${P^x}$*${{(1-p)}^{n-x}}$\\
        
        \begin{enumerate}
	        \item Probability of having 4 heads and amount Rs=4 is 
		%       \begin{align}
			\pr{X = 4,4,0.5} =${\Mycomb[4]{4}}$*${0.5^4}$*${{(1-0.5)}^{4-4}}$\\
		        =\[\frac{1}{16}\] 
		%	\end{align}
		\item Probability of having 3 heads and amount Rs=1.5 is
		       \pr{X = 3,4,0.5} =${\Mycomb[4]{3}}$*${0.5^3}$*${{(1-0.5)}^{4-3}}$\\
		        =\[\frac{1}{4}\] 
		\item Probablity  of  having 2 heads and amount Rs=-1 is 
		       \pr{X = 2,4,0.5} =${\Mycomb[4]{2}}$*${0.5^2}$*${{(1-0.5)}^{4-2}}$\\
		        =\[\frac{3}{8}\] 
		\item Probability  of  having 1 head and amount Rs=-3.5 is
		       \pr{X = 1,4,0.5} =${\Mycomb[4]{1}}$*${0.5^1}$*${{(1-0.5)}^{4-1}}$\\
		        =\[\frac{1}{4}\] 
		\item Probability  of  having 0 head and amount Rs=-6 is 
		       \pr{X = 0,4,0.5} =${\Mycomb[4]{0}}$*${0.5^4}$*${{(1-0.5)}^{4-0}}$\\
		        =\[\frac{1}{16}\] 
	\end{enumerate}
       \begin{center}
\begin{tabular}{||c c c c||} 
 \hline
 sr.no: & parameter & value & description \\ [0.5ex] 
 \hline\hline
 1 & $H_{0}$ & Rs 4 & 0 heads \\ 
 \hline
 2 & $H_{1}$ & Rs 1.5 & 1 heads \\
 \hline
 3 & $H_{2}$ & Rs -1 & 2 heads \\
 \hline
 4 & $H_{3}$ & Rs -3.5 & 3 heads \\
 \hline
 5 & $H_{4}$ & Rs -6 & 4 heads \\ [1ex] 
 \hline
\end{tabular}
\end{center}
         
	

\end{document}

